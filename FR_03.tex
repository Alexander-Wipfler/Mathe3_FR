\documentclass[t,14pt]{beamer}

%*****************************************************************************%
\usepackage{Styles/stylesGeneral_JR01}
\usepackage{Styles/stylesPresentation_JR01}
\usepackage{german}
\usepackage{amsmath}                                                          	% Zurücksetzen der Tabellen- und Abbildungsnummerierung je Sektion
\usepackage{amssymb}															% erforderlich für mathbb Definitionsbereich Reeele Zahlen...



\usepackage{babelbib}

\usepackage{stackrel}
\usepackage{array}

\usepackage{trfsigns}

\usepackage{tikz}
\usepackage{circuitikz}
\usetikzlibrary{shapes,arrows,arrows.meta,decorations,decorations.pathmorphing,decorations.pathreplacing,patterns,matrix}
\usetikzlibrary{babel,angles,quotes,calc,intersections}



%\usepgfplotslibrary{external} 
%\tikzexternalize[prefix=temp_graphics/]

\usepackage{geometry}
\geometry{
    paperwidth=210mm,
    paperheight=297mm,
    portrait,
%    left=25mm,
%    right=20mm,
%    top=10mm,
%    bottom=20mm
}


\usepackage[framemethod=default]{mdframed}
\usepackage{hyperref}




\newcommand*\diff{\mathop{}\!\mathrm{d\hspace{-1pt}}}	% Differentialzeichen
\newcommand*\Diff[1]{\mathop{}\!\mathrm{d^#1}} % Differentialzeichen höherer Ableitung
\newcommand*\jj{\mathop{}\!\mathrm{j}}	% Komplexe Zahl j
\newcommand*\euler{\mathrm{e}} % Zahl e
\newcommand{\vek}[1]{\textit{\textbf{#1}}}
\newcommand{\vekgr}[1]{\mathit{\mathbf{#1}}}
\newcommand{\Rg}{\textnormal{Rg}}
\newcommand{\und}{\quad\textnormal{und}\quad}
\newcommand{\mitm}{\quad\textnormal{mit}\quad}
\newcommand{\fuer}{\quad\textnormal{für}\quad}
\newcommand{\sonst}{\quad\textnormal{sonst}\quad}



% Copyright 2012 by Aécio S. R. Santos <aecio.solando@gmail.com>.
%
% In principle, this file can be redistributed and/or modified under
% the terms of the GNU Public License, version 2.
%
% However, this file is supposed to be a template to be modified
% for your own needs. For this reason, if you use this file as a
% template and not specifically distribute it as part of a another
% package/program, I grant the extra permission to freely copy and
% modify this file as you see fit and even to delete this copyright
% notice.

% Redefines the font type
\usepackage{helvet}
\usefonttheme[onlymath]{serif}




% Some useful colors
\definecolor{titleblue}{rgb}{0.1, 0.1, 0.7}
\definecolor{titlegreen}{rgb}{0.1, 0.7, 0.1}

% BLUE color scheme
\newcommand{\setcolorschemeblue}{
 \definecolor{titlecolor}{rgb}{0.1, 0.1, 0.7}
 \definecolor{bulletscolor}{rgb}{0.2, 0.2, 0.6}
 \definecolor{alertcolor}{rgb}{0.1, 0.7, 0.1}
}

% PURPLE color scheme
\newcommand{\setcolorschemepurple}{
 \definecolor{titlecolor}{rgb}{0.35, 0.2, 0.55}
 \definecolor{bulletscolor}{rgb}{0.44, 0.4, 0.70}
 \definecolor{alertcolor}{rgb}{0.35, 0.2, 0.55}
}

% GREEN color scheme
\newcommand{\setcolorschemegreen}{
 \definecolor{titlecolor}{rgb}{0, 0.5, 0.48}
 \definecolor{bulletscolor}{rgb}{0.2,0.2,0.7}
 \definecolor{alertcolor}{rgb}{.0, .68, .84}
}

% Define default colors scheme
\setcolorschemeblue

% Define color of alert text
\setbeamercolor{alerted text}{fg=alertcolor}

% block environment
%\setbeamertemplate{blocks}[rounded=true, shadow=true]
\setbeamertemplate{blocks}[rounded][shadow=true]
\setbeamercolor*{block body}{fg=black,bg=titlecolor!20!white}
%\setbeamercolor*{block title}{fg=titlecolor!70!black,bg=titlecolor!40!white}
\setbeamercolor*{block title}{fg=titlecolor!10!white,bg=titlecolor!75!white}
%\setbeamerfont{block title}{size=\large,series=\bf}

% Define font sizes
%\setbeamerfont{frametitle}{parent=structure,size=\Large}
%\setbeamerfont{framesubtitle}{parent=frametitle,size=\footnotesize}
%\setbeamerfont{itemize/enumerate body}{size=\fontsize{16pt}{17.6pt}}
%\setbeamerfont{itemize/enumerate body}{size=\fontsize{14pt}{15.4pt}}
%\setbeamerfont{itemize/enumerate subbody}{size=\fontsize{14pt}{15.4pt}}
%\setbeamerfont{itemize/enumerate subsubbody}{size=\footnotesize}

% Define font sizes for bibliography
\setbeamerfont{bibliography entry author}{size=\small}
\setbeamerfont{bibliography entry title}{size=\small}
\setbeamerfont{bibliography entry location}{size=\small}
\setbeamerfont{bibliography entry note}{size=\small}

% Redefine the cover title fonts
\setbeamerfont{title}{size=\LARGE, series=\bfseries}
\setbeamercolor{title}{fg=titlecolor}
% Redefine the title fonts
\setbeamerfont{frametitle}{size=\Large, series=\bfseries}
\setbeamercolor{frametitle}{fg=titlecolor}

% Uncomment to redefine bullets with round format
\useinnertheme[shadow]{rounded}
\setbeamertemplate{blocks}[rounded][shadow=\beamer@themerounded@shadow]
\setbeamertemplate{items}[ball]

% Redefine table of content colors
\setbeamercolor{section in toc}{fg=titlecolor}
\setbeamercolor{subsection in toc}{fg=titlecolor!40!black}

% Redefine bibliography colors
\setbeamercolor{bibliography entry author}{fg=titlecolor!25!black}
\setbeamercolor{bibliography entry title}{fg=titlecolor}
\setbeamercolor{bibliography entry location}{fg=titlecolor!25!black}

% Redefine bullets color
\setbeamercolor*{item}{fg=bulletscolor}

% Suppress figure caption name
\setbeamertemplate{caption}{\raggedright\insertcaption\par}

% Redefine spacing of left margin of bullets
\setlength{\leftmargini}{1.3em}
\setlength{\leftmarginii}{2em}
\setlength{\leftmarginiii}{2em}

% Redefine space between items in 'itemize' enviroment
\newlength{\wideitemsep}
\setlength{\wideitemsep}{\itemsep}
\addtolength{\wideitemsep}{0.25pt}
\let\olditem\item
\renewcommand{\item}{\setlength{\itemsep}{\wideitemsep}\olditem}

% Redefine space before a nested itemize
\makeatletter
\def\@listii{\leftmargin\leftmarginii
              \topsep    0.9ex
              \parsep    0\p@   \@plus\p@
              \itemsep   \parsep}
\def\@listii{\leftmargin\leftmarginiii
              \topsep    0.9ex
              \parsep    0\p@   \@plus\p@
              \itemsep   \parsep}
\makeatother


% Redefine width of text area margins
\setbeamersize{text margin left=1.3em,text margin right=1.3em}

% Define summary items depth
\setcounter{tocdepth}{2}

% Redefine styles of frames' title
\setbeamertemplate{frametitle} {
  \vspace{0.2cm}
  \ifbeamercolorempty[bg]{frametitle}{}{\nointerlineskip}%
  \begin{beamercolorbox}[]{frametitle}
    \ifbeamercolorempty[bg]{frametitle}{}{\nointerlineskip}%
    \usebeamerfont{frametitle}{%
      \strut\insertframetitle\strut\par%
    }
    {%
      \ifx\insertframesubtitle\@empty%
      \else
  \usebeamerfont{framesubtitle}\usebeamercolor[fg]{framesubtitle}\insertframesubtitle\strut\par
      \fi
      \vspace{-0.85cm}%
      {
  \textcolor{titlecolor!99!black}{\rule[7pt]{\linewidth}{1.2pt}\vspace{-4pt}}
      }
    }%
    \vskip-0.5ex%
    \if@tempswa\else\vskip-0.9cm\fi
  \end{beamercolorbox}%
  \vspace{0.1cm}
}

% Removes navigation bar
\beamertemplatenavigationsymbolsempty

% Redefine footline to show only slide number
\setbeamertemplate{footline}{
  \begin{beamercolorbox}[wd=1\paperwidth,ht=2.25ex,dp=1ex,right]{date in head/foot}%
  \textcolor{titleblue!99!black}{\rule[5pt]{\linewidth}{0.5pt}\vspace{-4pt}}
    \inserttitle ~ \insertdate \hfill 
    \insertauthor \hfill
    S. \insertframenumber{} von \inserttotalframenumber  % Current slide number and total of slides
    \hspace{2ex}
  \end{beamercolorbox}
}






% Table of contents popping up at the beginning of each subsection
%\AtBeginSubsection[]
%{
%  \begin{frame}<beamer>{Outline}
%    \tableofcontents[currentsection,currentsubsection]
%  \end{frame}
%}

\AtBeginSection[]
{
  \begin{frame}
   \begin{center}
    \Huge\color{titlecolor}\textbf{\vspace{.5cm}\\ Abschnitt \insertsectionnumber\\ \vspace{1cm}  \insertsection}
   \end{center}
  \end{frame}
}

% If you wish to uncover everything in a step-wise fashion, uncomment
% the following command: 
%\beamerdefaultoverlayspecification{<+->}


%*****************************************************************************%
\title[]{Vorlesung Mathematik 3}
\subtitle{Fouriertransformation}
\author[]{Prof. Dr. A. Wipfler\\ E-Mail: \textcolor{blue}{wipfler@dhbw-ravensburg.de}}
\institute[]{{\footnotesize Duale Hochschule Baden-W\"urttemberg, Ravensburg/Friedrichshafen}}
\date[]{Ausgabestand: \today}
\subject{Vorlesung Mathematik 3}

%\pgfdeclareimage[height=0.5cm]{university-logo}{Styles/DHBW_Logo_lang_2022.png}
%\logo{\pgfuseimage{university-logo}}

%*****************************************************************************%
\begin{document}

%-----------------------------------------------------------------------------%
\begin{frame}
\titlepage
\end{frame}





%-----------------------------------------------------------------------------%
\begin{frame}{Inhalt der Vorlesung}


Dieser Vorlesungsteil widmet sich den Fouriertransformationen. Diese erweiteren das Konzept der Fourierreihe
auf transiente Funktionen, also solche, die nicht periodisch sind.
\begin{enumerate}
\item Periodendauer $T\to\infty$
\item Spaltfunktion
\item Faltung
\item Differenziation und Integration
\item Verschiebung
\end{enumerate}

\end{frame}

\section{Transiente Funktionen}
\begin{frame}{\thesection .) $T\to\infty$}
\begin{itemize}
    \item \emph{Komplexe Fourierreihe}
    \begin{align*}
        f(t)    &= \suml_{k=-\infty}^{\infty}c_k\euler^{jk\omega_0t}\\
        c_k     &= \frac{1}{T}\intl_{-\frac{T}{2}}^{\frac{T}{2}}f(t)\euler^{-jk\omega_0t}\diff t
    \end{align*} 
    \item Betrachtung für $T\to\infty$ unter Berücksichtigung von 
    \[
    \omega_0 = \frac{2\pi}{T}
    \]
    \begin{center}
    \begin{tikzpicture}[scale = 1.4]
    \draw [step = .5cm,gray] (0,0) grid (13,10);
    \end{tikzpicture}
    \end{center}
\end{itemize}
\begin{alertblock}{Fouriertransformation}
    Die \emph{Fouriertransformation} ordnet einer Funktion $f(t)$ im Zeitbereich eine Funktion $F(j\omega)$ im 
    Frequenzbereich zu.
    \begin{align*}
        F(j\omega)  &= \intl_{-\infty}^{\infty} f(t)\euler^{-j\omega t}\diff t                          &\text{Hintransformation}\\
        f(t)        &= \frac{1}{2\pi}\intl_{-\infty}^{\infty}F(j\omega)\euler^{j\omega t}\diff\omega    &\text{Rücktransformation}
    \end{align*}
\end{alertblock}
\end{frame}



\section{Spaltfunktion}
\begin{frame}{\thesection .) Spaltfunktion}
\begin{center}
    \begin{tikzpicture}[>=latex]
        %--- Einlaufende Welle
        \foreach \x in {-4,-3,-2,-1}
        \draw [blue, thick] (\x,-2) -- (\x,2);
        \draw [<->,blue] (-2,0) -- (-1,0) node [midway,anchor=south]{$\lambda$};
        %--- Spalt
        \draw [gray,->] (0,0) -- (0,.8) node [anchor = south east] {$x$}; 
        \draw [very thick] (0,3) -- (0,1) node [anchor = west]{$\frac{D}{2}$}  (0,-1) node [anchor = west]{$-\frac{D}{2}$} -- (0,-3);
        %--- Optische Achse
        \draw [gray,dashed] (0,0) -- (5,0);
        %--- Schirm
        \draw [very thick] (5,-3) -- (5,3);
        \draw [gray,->] (5.1,0) -- (5.1,2) node [anchor = west] {$\xi$};
        %--- Beugungsmuster
        \draw[thick, blue, domain = -3:-.0005, variable = \t,samples = 200] plot ({6+2*(sin(5*\t r)/(5*\t))},\t);
        \draw[thick, blue, domain = .0005:3, variable = \t,samples = 200] plot ({6+2*(sin(5*\t r)/(5*\t))},\t);
    \end{tikzpicture}
\end{center}
\begin{center}
    \begin{tikzpicture}[scale = 1.4]
    \draw [step = .5cm,gray] (0,0) grid (13,12);
    \end{tikzpicture}
\end{center}
\begin{definition}
    Die \emph{Spaltfunktion} (alternativ: \emph{Sinus cardinalis}) ist definiert als 
    \[
    \sinc (x)=\frac{\sin \pi x}{\pi x}
    \]
\end{definition}
\end{frame}


\section{Faltung}
\begin{frame}{\thesection .) Faltung}
    \begin{definition}
        Die \emph{Faltung} zweier Funktionen $f(t)$ und $g(t)$ ist definiert als:
        \[
        f(t)*g(t) = g(t)*f(t) = \intl_{-\infty}^{\infty} f(\tau)g(t-\tau)\diff \tau
        \]
    \end{definition}
    \begin{itemize}
        \item Bestimmung der Fouriertransformation von $f(t)*g(t)$
        \begin{center}
            \begin{tikzpicture}[scale = 1.4]
            \draw [step = .5cm,gray] (0,0) grid (13,12);
            \end{tikzpicture}
        \end{center}
    \end{itemize}
    \begin{alertblock}{Faltungssatz}
        Die Fouriertransformierte einer Faltung zweier Funktionen $f$ und $g$ entspricht dem \emph{Produkt} der beiden Fouriertransformierten
        $F$ und $G$ der einzelnen Funktionen:
        \begin{align*}
            f\quad   &\laplace\quad F\\
            g\quad   &\laplace\quad G\\
            f*g\quad &\laplace\quad FG
        \end{align*}
    \end{alertblock}
\end{frame}

\begin{frame}
    \begin{itemize}
        \item \alert{Beispiel:} 
        \begin{align*}
            f(t)    &= \left\{\begin{array}{l}
                1 \fuer -1\leq t \leq 2\\
                0 \sonst
            \end{array}  \right.\\
            g(t)    &= \left\{\begin{array}{l}
                1 \fuer -2 \leq t \leq 1\\
                0 \sonst
            \end{array} \right.
        \end{align*}
        \begin{center}
            \begin{tikzpicture}[>=latex]
                %--- Koordinatensystem
                \draw [->] (-3,0)  -- (3.2,0)   node [anchor = north west]{$t$};
                \draw [->] (0,-.2) -- (0,1.4) node [anchor = south east]{$f,g$};
                %--- Ticks
                \foreach \x in {-3,-2,...,3}
                \draw (\x,0) -- (\x,-.2) node [anchor = north, font = \small] {$\x$};
                %--- f(t)
                \draw [red, thick] (-3,0) -- (-1,0) -- (-1,1) -- (2,1)node [anchor = south] {$f(t)$} -- (2,0) -- (2.8,0);
                 %--- g(t)
                 \draw [blue, thick] (-3,0) -- (-2,0) -- (-2,1) node [anchor = south] {$g(t)$} -- (1,1) -- (1,0) -- (2.8,0);
            \end{tikzpicture}
        \end{center}
    \end{itemize}
\end{frame}

\section{Differenziation und Integration}


\section{Verschiebung}



%\begin{center}
%\begin{tikzpicture}[scale = 1.4]
%\draw [step = .5cm,gray] (0,0) grid (13,14);
%\end{tikzpicture}
%\end{center}


\end{document}